
\chapter{Introdução}
\label{ch:intro}
O petróleo é uma das principais fontes de combustível, representando cerca de um terço do total do consumo de combustível mundial. Em 2016, o consumo global desse combustível aumentou 1,6\% em relação a 2015, totalizando em 96,6 milhões de barris de petróleo por dia~\cite{BritishPetroleum2017, ANP2017}. No entanto, para o mesmo período, a produção global de petróleo não cresceu na mesma velocidade. Em 2016 foram produzidos 92,2 milhões de barris de petróleo por dia e, apesar de um aumento de 0,4\% em relação a 2015, a produção foi aquém da demanda.

Em virtude disso é importante buscar por formas de otimizar a produção de petróleo para suprir toda a demanda do mercado. A produção desse combustível não é uma tarefa fácil. Desenvolver e gerenciar campos de petróleo é uma atividade de risco, exige altos investimentos e traz uma série de desafios. Para que o produto chegue até o mercado há uma série de etapas que vão desde a descoberta e aquisição de dados sobre o reservatório de petróleo, a definição de uma estratégia de produção para a exploração do campo e o processamento e refinamento do óleo\cite{ANP2017}.

Dentre essas etapas, definir a estratégia de produção é um passo importante para explorar de forma eficiente o campo de petróleo. Uma vez definida, espera-se que tal estratégia seja utilizada por todo o período de exploração do campo. De forma sucinta, a definição da uma estratégia de produção (DEP) pode ser modelada como um problema de otimização, cujo objetivo final é definir e dimensionar o sistema de produção a ser adotado em um campo de petróleo de forma a maximizar algum critério. Tal critério, na maioria das vezes, é o Valor Presente Líquido (VPL), que é, essencialmente, o lucro obtido ao final da exploração do campo \cite{Marques2012}.

Dimensionar o sistema de produção para a exploração do campo também não é uma tarefa simples devido ao grande número de variáveis, relacionadas às características do sistema de produção, e às incertezas associadas ao problema e que dizem respeito às propriedades físicas do reservatório, à dinâmica do mercado e aos eventuais problemas com os equipamentos. Além disso, o próprio modelo que descreve o campo contém incertezas significativas, dado que é uma tentativa de representar algo que se encontra a quilômetros de profundidade.

Apesar de ser possível utilizar ferramentas clássicas para resolver esse problema de otimização, como Programação Inteira Mista \cite{Rosenwald1974} ou métodos baseados em gradiente \cite{Bangerth2006, Handels2007}, para que tais métodos funcionem diversas simplificações devem ser consideradas, o que pode levar a soluções que não condizem com a realidade ou, ainda, corre-se o risco do algoritmo ficar preso em uma solução de ótimo local (Nasrabadi, Morales and Zhu, 2012)(Nasrabadi, Morales and Zhu, 2012). Dito isso, é necessário recorrer a ferramentas alternativas para obtenção de boas soluções para o problema de DEP como, por exemplo, as meta-heurísticas \cite{Blum2003}(Blum \& Roli, 2003). Meta-heurísticas são ferramentas de propósito geral, ou seja, ferramentas que podem ser aplicadas a diferentes problemas sem a necessidade de grandes adaptações, independentemente das características intrínsecas do problema em questão (convexidade, diferenciabilidade, multimodalidade etc.).  Dentre as meta-heurísticas, cabe destacar as meta-heurísticas populacionais, como os algoritmos genéticos, que atuam em cima de uma população de soluções candidatas para o problema, refinando-as sucessivamente até que uma solução suficientemente boa seja encontrada \cite{Back2000, decastro2006-CRC-fundamentals}(BÄCK ET AL., 2000; DE CASTRO, 2006). Além disso, variantes de tais ferramentas buscam manter diversidade entre as soluções candidatas na população, o que as tornam menos suscetíveis aos ótimos locais do problema e capazes de retornar, em uma única execução, múltiplas soluções distintas e de boa qualidade \cite{DeFranca2010}(DE FRANÇA ET AL. 2010).

Muitas dessas características fazem com que as meta-heurísticas tenham grande potencial de aplicação ao problema de DEP, principalmente quando combinadas aos resultados gerados por simuladores de campos de petróleo e por outras ferramentas auxiliares que retornam, por exemplo, o VPL associado a uma dada estratégia de produção. Nesse contexto, as meta-heurísticas são responsáveis por buscar estratégias de produção, ou seja, definir os parâmetros do sistema de produção, enquanto que os módulos auxiliares são responsáveis por avaliar tais estratégias.

Por outro lado, o fato de serem ferramentas de propósito geral faz com que as meta-heurísticas tendam a demandar um número relativamente alto de avaliações de soluções candidatas até que uma solução de boa qualidade seja encontrada e retornada. Não é incomum encontrar, na literatura, aplicações de meta-heurísticas que envolvam dezenas de milhares de avaliações de soluções candidatas durante o processo de otimização \cite{Coelho2010}(Coelho \& Von Zuben, 2010). A exigência de um alto número de avaliações de função é crítica para a aplicação de meta-heurísticas a problemas de DEP. Em tais situações, cada avaliação de uma solução candidata exige uma execução do software de simulação do campo de petróleo, que pode levar de alguns minutos a várias horas dependendo do nível de detalhamento do modelo do campo em estudo. 

Sendo assim, esse trabalho tem como objetivo estudar formas de aplicação de meta-heurísticas ao problema de definição de estratégias de produção (DEP) em campos de petróleo com o intuito de diminuir o custo computacional necessário para que o algoritmo encontre uma solução de boa qualidade. Para tal, dentre as meta-heurísticas disponíveis na literatura, foram escolhidos os algoritmos genéticos para serem aplicados ao problema de DEP. Aqui são propostos operadores de busca específicos para o problema de DEP e uma versão modificada de algoritmo genético. Para que tais objetivos pudessem ser atingidos foram realizados estudos sobre as principais meta-heurísticas disponíveis na literatura para tratar problemas de otimização e sobre gerenciamento de campos de petróleo, para identificar características do problema de DEP que podem auxiliar no processo de otimização. O algoritmo genético e os operadores propostos foram utilizados para otimizar o posicionamento de poços de uma estratégia de produção em um reservatório de petróleo sintético, sendo que tanto o escopo do problema quanto o modelo utilizado foram definidos em conjunto com os engenheiros do grupo UNISIM\footnote{UNISIM é um grupo de pesquisa em simulações numéricas para reservatórios de petróleo. O grupo pertence ao Departamento de Energia, Divisão de Engenharia de Petróleo, da Faculdade de Engenharia Mecânica da UNICAMP e ao Centro de Estudos de Petróleo. Mais informações em: www.unisim.cepetro.unicamp.br/br/}.

Este documento está estruturado da seguinte forma: o Capítulo ~\ref{ch:intro}., que se seguiu, contextualizou o trabalho e apresentou suas motivações e objetivos; os conceitos teóricos e a revisão bibliográfica sobre Meta-heurísticas e Gerenciamento de Campos de Petróleo são discutidos no Capítulo 2; os operadores desenvolvidos aqui, bem como as modificações realizadas ao Algoritmo Genético são apresentados no Capítulo 3; no Capítulo 4 são descritas as ferramentas computacionais utilizadas para o desenvolvimento desse trabalho, bem como o caso de estudo e a estrutura dos experimentos realizados; os resultados obtidos são discutidos no Capítulo 5; e, por fim, as conclusões são comentadas no Capítulo 6.